\documentclass[english,12pt,a4paper]{article}
\usepackage[T1]{fontenc}
\usepackage{babel}
\usepackage{hyperref}
\usepackage{ffcode}
\hypersetup{
	colorlinks=true,
	linkcolor=black,
	urlcolor=blue,
}
\usepackage{xcolor}
\definecolor{light-gray}{gray}{0.95}
\newcommand{\code}[1]{\colorbox{light-gray}{\texttt{#1}}}

\title{PKM 4 Documentation}
\author{Gordinator}
\begin{document}
	\maketitle
	\tableofcontents
	\section{Introduction}
	This Document contains information about PKM 4, how it works, and how to implement it.
	\subsection{What Is PKM4?}
	PKM is Pythinux's package manager. Version 4 released with Pythinux 3.0.
	\subsection{What Is Pythinux?}
	Pythinux is a multi-user single- operating environment written in Python and designed in a way similar to UNIX-like systems. 
	\section{Terminology}
	\begin{itemize}
		\item \textbf{SZIPS}: \textbf{S}tandard \textbf{Z}ipped \textbf{I}nternal \textbf{P}rogram \textbf{S}cheme. Refers to the series of file formats for packaging Pythinux programs.
		\item \textbf{SZIP4}: Version 4 of SZIPS. Current version.
		\item \textbf{Repository}: An Internet location which stores programs.
		\item \textbf{buildr}: Utility that creates repositories.
	\end{itemize}
	\section{Repositories}
	\subsection{Hosting}
	Repository contents are fetched over HTTP. As such, any HTTP server is suitable works.
	\subsection{Folder Structure}
	\begin{ffcode}
		./buildr.py
		./db.ini
		./<program>
		./<program>/program.szip4
		./<program>/program/program.ini
		./<program>/program/*
	\end{ffcode}
\end{document}